version https://git-lfs.github.com/spec/v1
oid sha256:031973c247668cb402ecf8e838afe9a63322ba3d3a290586c842644a49bdfd02
size 1611
